\begingroup
\let\clearpage\relax
\let\cleardoublepage\relax
\let\cleardoublepage\relax

\chapter*{Summary} % Abstract name
\addtocontents{toc}{\protect\vspace{\beforebibskip}} % Place the bibliography slightly below the rest of the document content in the table of contents
\addcontentsline{toc}{chapter}{\tocEntry{Summary}}

The brain can be conceived as a complex system,
as it is made up of nested networks of interactions
and moreover,
demonstrates emergent-like behaviors such as oscillations.
Based on this conceptualization,
various tools and frameworks that stem from the field of complex systems
have been adapted to answer neuroscientific questions.
Certainly, using such tools for neuroscientific questions has been insightful for understanding the brain as a complex system.
Nevertheless,
they encounter limitations when they are adapted for the purpose of understanding the brain,
or perhaps better should be stated that,
developing approaches which are closer to the neuroscience side can also be instrumental for approaching the brain as a complex system.

\marginpar{\autoref{cha:brain-as-complex}}

In this thesis, after an elaboration on the motivation of this endeavor in \autoref{cha:brain-as-complex},
we introduce a set of complementary approaches,
with the rationale of exploiting the development in the field of systems neuroscience in order to
be close to the neuroscience side of the problem,
but also still remain connected to the complex systems perspective.
Such complementary approaches can be envisioned through different apertures.
In this thesis, we introduce our complementary approaches,
through the following apertures:
neural data analysis (\autoref{cha:appr-thro-nda}),
neural theories (\autoref{cha:appr-thro-theo}), and
cognition (\autoref{cha:appr-thro-behav}).

In \autoref{cha:appr-thro-nda}, we argue that multi-scale and cross-scale analysis of neural data is one of the important aspects of the neural data analysis from the complex systems perspective toward the brain.
Furthermore, we also elaborate that,
investigating the brain across scales, is not only important from the abstract perspective of complex systems,
but also motivating based on a variety of empirical evidence on coupling between brain activity at different scales, neural coordination and theoretical speculations on neural computation.
\marginpar{\autoref{cha:appr-thro-nda} \\
  \autoref{cha:paper-5}/\nameref{cha:paper-5} \\
  \autoref{cha:paper-gpla}/\nameref{cha:paper-gpla}  \\
  \autoref{cha:paper-besserve2020ned}/\nameref{cha:paper-besserve2020ned}
}
Based on this motivation we first very briefly discuss some of the relevant cross-scale neural data 
analysis methodologies and then introduce two novel methodologies that have been developed as parts of this thesis
(\nameref{cha:paper-5}, \nameref{cha:paper-gpla}, and \nameref{cha:paper-besserve2020ned}).
In \nameref{cha:paper-5} and \nameref{cha:paper-gpla} we introduced a multi-variate methodology for investigating spike-LFP relationship and in \nameref{cha:paper-besserve2020ned} we introduced a methdology for detecting cooperative neural activities (neural events) in local field potentials,
that can be used as a trigger to investigate simultanious activity in larger and smaller scales.
A prominent example of these neural events are sharp wave-ripples that has been shown to co-occur with precise coordination in the spiking activity of individual neurons and the large-scale brain activity as well.


In \autoref{cha:appr-thro-theo}, we introduce a new aperture through neural theories.
One way of approaching the brain as a complex system is seeking for connections between theoretical frameworks that stem from the field of complex systems and the ones established in neuroscience.
On the complex systems side, we consider the \emph{criticality hypothesis of the brain} that has strong roots in the field of complex systems, and on the neuroscience side, we consider the \emph{efficient coding} which is one of the most important theoretical frameworks in systems neuroscience.
We first briefly introduce the background on efficient coding and criticality,
and elaborate further on the motivation behind our integrative approach.
In \nameref{cha:paper-safavi2020cribay}, we present our interim results,
\marginpar{\autoref{cha:appr-thro-theo} \\
  \autoref{cha:paper-safavi2020cribay}/\nameref{cha:paper-safavi2020cribay}
}
which suggests the two influential, and previously disparate fields -- efficient coding, and criticality -- might be intimately related.
We observed that, in the vicinity of the parameters that leads to optimized performance of a network implementing neural coding,
the distribution of avalanche sizes follow a power-law distribution.
In \nameref{cha:paper-safavi2020cribay} we also provide an extensive discussion on the implication of our interim results and its future extensions.
Moreover, in \nameref{cha:paper-safavi2020cribay} we also introduce another perspective which motivates such investigations,
namely seeking for potential bridges between \emph{neural computation} and \emph{neural dynamics}.

In \autoref{cha:appr-thro-behav}, we argue that binocular rivalry,
as a key phenomenon to investigate consciousness,
is particularly relevant for a complex systems perspective toward the brain.
Based on this insight,
we suggest and conduct novel experimental work,
namely, studying this phenomenon at a mesoscopic scale, that has not been done before.
Surprisingly, in the last 30 years, almost all the previous studies on binocular rivalry were either focused on micro-scale (level of an individual neuron) or the macro-scale (level of the whole brain).
Therefore, our work in this domain not only is valuable from the perspective of complex systems,
but also for understanding the neural correlate of visual awareness \emph{per se}.
In \nameref{cha:paper-safavi2014}, \nameref{cha:paper-safavi2018}, \nameref{cha:paper-kapoor2020}, and \nameref{cha:paper-dwarakanath2020} we elaborate on the outcome of this investigation.
\marginpar{\autoref{cha:appr-thro-behav} \\
  \autoref{cha:paper-safavi2014}/\nameref{cha:paper-safavi2014} \\
  \autoref{cha:paper-safavi2018}/\nameref{cha:paper-safavi2018}  \\
  \autoref{cha:paper-kapoor2020}/\nameref{cha:paper-kapoor2020}  \\
  \autoref{cha:paper-dwarakanath2020}/\nameref{cha:paper-dwarakanath2020}
}
\nameref{cha:paper-safavi2014} and \nameref{cha:paper-safavi2018} were prerequisite for the binocular rivalry experiments.
In \nameref{cha:paper-safavi2014} we elaborate on the importance of studying prefrontal cortex (PFC)
(which was the region of interest in our investigation)
for understating the neural correlate of visual awareness.
In \nameref{cha:paper-safavi2018} we investigate
the basic aspects of neural responses (tuning curves and noise correlations) of PFC units to simple visual stimulation
(in a similar setting used for our binocular rivalry experiments).
In \nameref{cha:paper-kapoor2020} and \nameref{cha:paper-dwarakanath2020} we investigate the neural correlate of visual awareness at a mesocopic scale
(which is motivating from the complex system perspective toward the brain).
We show that content of visual awareness is decodable from the population activity of PFC neurons (\nameref{cha:paper-kapoor2020})
and show oscillatory dynamics of PFC (as a reflection of collective neural activity) can be a relevant signature for perceptual switches (\nameref{cha:paper-dwarakanath2020}).
I believe that this is just the very first step toward establishing a connection from a complex systems perspective to cognition and behavior.
Various theoretical and experimental steps need to be taken in the future studies to build a solid bridge between cognition and complex systems perspective toward the brain.

The last chapter, \autoref{cha:brain-as-complex-adaptive}, is dedicated to an outlook, a subjective perspective on how this research line can be proceeded.
In the spirit of this thesis which is \emph{searching for principles},
I believe we are missing an important aspect of the brain which is its \emph{adaptivity}.
At the end, brain, even the most ``complex system'', needs to survive in the environment.
Indeed, in the field of \emph{complex adaptive systems}, the intention is understanding very similar 
\marginpar{\autoref{cha:brain-as-complex-adaptive}}
questions in the nature.
Inspired by ideas discussed in the field of complex adaptive systems,
I introduce a set of new research directions which intend to  incorporate the adaptivity aspect of the brain as one of the principles. 
These research directions also remain close to the neuroscience side, similar to the intention of the research presented in this thesis.



\endgroup			

\vfill


%%% Local Variables:
%%% mode: latex
%%% TeX-master: "../phdThesis_csb"
%%% End:


