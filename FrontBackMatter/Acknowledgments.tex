% Acknowledgements

%\pdfbookmark[1]{Acknowledgements}{Acknowledgements} % Bookmark name visible in a PDF viewer

\begin{flushright}{\slshape    
The most beautiful aspect of science is that it is a collaborative enterprise} % \\ \medskip
--- Freeman J. Dyson
\end{flushright}

\bigskip

%----------------------------------------------------------------------------------------

\begingroup

\let\clearpage\relax
\let\cleardoublepage\relax
\let\cleardoublepage\relax
\chapter*{Acknowledgment} % Acknowledgements section text
\addcontentsline{toc}{chapter}{\tocEntry{Acknowledgements}}


Science is a \emph{collaborative endeavor}
% \footnote{Science should be a \emph{collaborative endeavor} rather a big completions; I hope we --scientist-- always have it mind!}
and this small piece of work could not have been accomplished without the help of many people.
Indeed, that's the reason I mostly mention \emph{we did} rather \emph{I did} in this thesis.
Probably I managed to mention a subset of those people here in this acknowledgment.
Writing the acknowledgment section was one of the most pleasant parts for me as it has the sign of the collective and cooperative attitude of humankind.
I hope I manage to reinforce this aspect of science, cultivate it, and ultimately do better science.


I'm grateful to Nikos Logothetis
for providing me
the freedom for exploration, designing my research, and sufficient resources to realize the ideas.
Doing the PhD in his inspiring lab was a unique opportunity for me (from multiple perspectives).
Perhaps, one of the most valuable experiences I had in his lab was doing experiments and theory (in the extreme of each) both during my PhD which already helped me to craft my neuroscientific vision and will help me throughout my career.
I wish the crises related to animal activists never happened and I could have more learning opportunities from him and the members of his lab.


I'm thankful to Michel Besserve,
with whom I pursued an important part of my PhD research in Nikos' lab.
I'm grateful for constructive discussions and help whenever and in whichever way he could provide 
besides his own intensive and resource-demanding research.


I'm grateful to Anna Levina [and all of her lab members], who has always welcomed me as her ``permanent guest'' in their lab and provided me with scientific and moral support whenever I asked.
Furthermore, I cannot imagine pursuing a scientific question smoother than how we started the project on an efficient coding criticality.
I literally walked into her office, explained the idea, and she provided whatever support needed for moving it forward (and being patient with its slow progress).
Matthew Chalk also joined us at the very beginning on Anna's call and supported generously and I'm very thankful to him as well.

I'm also grateful to all members of my PhD advisory board and PhD committee for sparing time for the meetings and baring with all  administrative works.
Indeed, all members were already mentioned above except Martin Giese,
who not only  helped me as a member of my PhD advisory board and PhD committee,
but also he was one of my greatest teachers during my neuroscience training in T\"ubingen;
and Gustabo Deco, who was not only was a member of my PhD committee, but also being  a constant source of inspiration;
and Sonja Gr\"un who not only was the external reviewer of my thesis, but also taught me much on developing statistical methods for neural data analysis by her rigorous methodological research.

I'm thankful
to Fanis (Theofanis) Panagiotaropoulos and his group whom I conducted all my experimental work with non-human primates.
In a similar vein, I'm also grateful to Vishal Kapoor, from whom I learn many unwritten tips and tricks for handling non-human primates.
Also, with Fanis and Vishal (and later on Abhilash Dwarakanath) I started extensively exploring and understanding the fascinating pheno\-menon of binocular rivalry.

All the staff at Max Planck Institute for Biological Cybernetics, department of physiology of cognitive processes were supportive and helpful.
In particular thankful to Conchy Moya who solely runs the administration of the lab.
No need to say, technicians of the Max Planck Institute for Biological Cybernetics
(including colleagues in workshops and animal facilities) were essential for conducting experiments,
and I'm grateful to them.
My special thanks to Joachim Werner for his \emph{literally} $24/7$ support for the lab.
Joachim helpfulness and kindness was beyond system administration support,
as a friend, he generously helped whenever I asked.
In a similar vein, I'm thankful to the Graduate Training Centre of Neuroscience and International Max Planck Research School for Cognitive and Systems Neuroscience for all they support,
in particular for being an important bridge to let me walk from my Physics world to the world of Neuroscience, and furthermore extensively explore this new field.


In the course of my research, I have approached many scientists to ask questions or even get consulting over multiple conversations.
Thanks Larissa Albantakis and Erick Hoel for helping me to understand the IIT and causal emergence;
thanks Afonso Bandeira and Asad Lodhia for multiple meetings they made to discuss issues related Random Matrix Theory (RMT);
thanks Uwe Ilg for his help for analysis of Optokinetic Nystagmus (OKN) responses of monkeys.

I'm thankful to many colleagues and friends for their generous scientific and moral support.
Roxana Zeraati for her indispensable scientific and moral support on countless occasions;
Yusuke Murayama for his altruistic support, especially moral one;
Behzad Tabibian for his pleasant company during the PhD (and beyond);
Hadi Hafizi,
Juan F Ramirez-Villegas and Kaidi Shao for their help in various stages of the PhD;
Daniel Zaldivar for sharing his valuable experiences about various stages of his scientific career.
Thanks to many colleagues at Max Planck campus, University of T\"ubingen and elsewhere
for creating a scientific and friendly atmosphere around me:
Ali Danish Zaidi,
Andre Marreiros,
Catherine Perrodin,
Fereshte Yousefi,
Franziska Br\"oker.
Georgios Keliris,
Hans Kersting,
Hamed Bahmani,
Hamid Ramezanpour,
Hao Mei,
Jennifer Smuda, 
Leili Rabbani,
Leonid Fedorov,
Maryam Faramarzi,
Mingyu Yang,
Mojtaba Soltanlou,
Oleg Vinogradov,
Parvaneh Adippour,
Parvin Nemati,
Reza Safari,
Sina Khajehabdollahi,
Tanguy Fardet,
Yiling Yang,
and NeNa organzier team of year 2014, 2015 and 2016.

During my PhD I was lucky to be part of multiple groups and communities.
The first one was the \href{https://www.opencon.community/}{OpenCon} community.
I'm thankful to all the people with whom I have interacted while being active in this community
(and still interacting occasionally);
in particular, Ali Ghaffaari, with whom we initiated developing the idea of \emph{d-index} in OpenCon 2018.
The second one was
founding the \href{http://www.pro-test-deutschland.de/}{Pro-Test Deutschland} together with multiple dedicated fellows.
During working with this dedicated group of people I realized the value, power, and importance of science communication.
Toward the end of my PhD, I had the chance to be one of the ombudspersons in Max Planck Institute for Biological Cybernetics and also PhD representatives in a different period.
These contributions taught me how critical is to care about mental health in academia.
Most importantly, I should say being part of these communities, beyond what I learned about open science, science communication, and mental health,
I recognized the power \emph{and the value} of collective actions by people whose fuel is pure motivation.


Last and not least (actually the most), I should express my gratitude to my extended family.
Just the presence of my old close friends, Amirhossein Ketabdar, Hadi Hafizi, Hadi Masrour, and Mohsen Soltani somewhere on the planet was already morale for me,
leaving aside their support whenever I asked!
I'm also grateful for new friendships after moving to Germany, Andriana Rina, and Manuel Alexantro.
Coming to family I should say, \emph{Ohana} means family,
family means nobody gets left behind or forgotten
(Lilo \& Stitch --- see the \href{https://www.youtube.com/watch?v=-U0xGBNl2fE}{video}).
I wish my father (who passed the way a long ago) also could read these lines and
wish I could  know his thoughts and feelings about the path of science that I chose for my life ... . 
I doubt that I can find the words to acknowledge the unconditional support and love of my partner,
nevertheless I guess the these synergistic music-poem:
\href{https://www.youtube.com/watch?v=HU9vCVTLuhM}{music 1a}/
\href{https://www.youtube.com/watch?v=xf4dRiLAis4}{music 1b} -
\href{http://www.nosokhan.com/library/Topic/125O}{poem 1}
and
\href{https://www.youtube.com/watch?v=pNzFf-tfc3k}{music 2a}/
\href{https://www.aparat.com/v/o8XUh/%D8%AD%D8%B3%DB%8C%D9%86_%D8%B9%D9%84%DB%8C%D8%B2%D8%A7%D8%AF%D9%87_%D9%88_%D9%85%D8%AD%D9%85%D8%AF_%D9%85%D8%B9%D8%AA%D9%85%D8%AF%DB%8C_-_%D8%A8%DB%8C%D8%A7_%D8%AA%D8%A7_%DA%AF%D9%84_%D8%A8%D8%B1%D8%A7%D9%81%D8%B4%D8%A7%D9%86%DB%8C%D9%85}{music 2b} -
\href{https://fa.wikisource.org/wiki/%D8%AF%DB%8C%D9%88%D8%A7%D9%86_%D8%AD%D8%A7%D9%81%D8%B8/%D8%A8%DB%8C%D8%A7_%D8%AA%D8%A7_%DA%AF%D9%84_%D8%A8%D8%B1_%D8%A7%D9%81%D8%B4%D8%A7%D9%86%DB%8C%D9%85_%D9%88_%D9%85%DB%8C_%D8%AF%D8%B1_%D8%B3%D8%A7%D8%BA%D8%B1_%D8%A7%D9%86%D8%AF%D8%A7%D8%B2%DB%8C%D9%85}{poem 2}
express [far better than my powerless words] how she has been helping me to get my/our bearings.



The words expressed here do not give full justice to all the support I have received and the adversity I have experienced.
In particular, during my PhD, the unpleasantness of the latter has dominated the former,
but it's not common to write about the latter in the "acknowledgment" section.
I wish I could write an articulated acknowledgment section that could reflect both pleasant and unpleasant sides in a fair manner,
but \dots!
Nevertheless,
perhaps related to my unpleasant experiences, I should at least mention that
some of the most valuable lessons I've learned during my PhD has been practicing being patient,
the dos and don'ts for conflict resolution, many coping skills.
Leaving out the non-zero probability of facing similar unpleasant situations,
conflicts and the like can still arise.
Therefore, learning these lessons will certainly be helpful.
That being said,
I should also be thankful for my adversaries that taught me all that.
Furthermore, by going through these adversities, I've realized  how severe lack of accountability, compassion and altruism can affect people and how easy is turning adversity to joy,
simply by having a little bit of accountability, compassion and altruism.
Finally, I hope this acknowledgment didn't undervalue the support of whom I've mentioned.
Even worse, I hope I have not forgotten the support of someone due to the lapses.
If that was the case, I truly apologize.

\endgroup


%%% Local Variables:
%%% mode: latex
%%% TeX-master: "../phdThesis_csb"
%%% End:
