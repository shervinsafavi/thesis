\chapter{Paper \rom{6}} \label{cha:paper-safavi2018}

\section*{Paper information} % 

\begin{description}
\item[Title:]
  Nonmonotonic spatial structure of interneuronal correlations in prefrontal microcircuits
\item[Authors:]
  Shervin Safavi$^*$,
  Abhilash Dwarakanath$^*$,
  Vishal Kapoor,
  Werner Joachim,
  Nicholas Hatsopoulos,
  Nikos K. Logothetis,
  Theofanis I. Panagiotaropoulos 
  ($^*$ indicate equal contributions) 
\item[Status:]
  Published in PNAS, see \citet{safaviNonmonotonicSpatialStructure2018}
\item[Presentation at scientific meetings:]
  NeNa 2015 \cite{dwarakanathTemporalRegimesStateDependent2015},
  AREADNE 2016 \cite{safaviNonMonotonicCorrelationStructure2016}
\item[Author contributions:]
  Conceptualization, T.I.P.; 
  Methodology, S.S., A.D., V.K. and T.I.P.;
  Software, S.S., A.D., T.I.P. and J.W.;
  Formal Analysis, S.S., A.D. and T.I.P.;
  Investigation, V.K., A.D., T.I.P., S.S. and N.G.H.;
  Resources, N.K.L.;
  Data Curation, A.D., T.I.P., V.K., and S.S.;
  Writing -- Original Draft, T.I.P., S.S., and A.D.;
  Writing -- Review \& Editing: V.K., A.D., T.I.P., N.G.H., and N.K.L.;
  Visualization, S.S., A.D, V.K. and T.I.P.;
  Supervision and Project administration, T.I.P.;
  Funding acquisition, N.K.L.
\end{description}

%%% Local Variables:
%%% mode: latex
%%% TeX-master: "../phdThesis_csb"
%%% End:


\section*{Summary} %
\subsection*{Motivation}

It has been suggested that mammalian's neocortex follow certain canonical features
\cite{douglasCanonicalMicrocircuitNeocortex1989,douglasNeuronalCircuitsNeocortex2004,douglasMappingMatrixWays2007,harrisCorticalConnectivitySensory2013}.
One of the features is in the spatial pattern of connectivity.
Indeed, there is a large body of evidence suggesting that functional connectivity, inferred based on spike count correlations \cite{cohenMeasuringInterpretingNeuronal2011},
rapidly decay as a function of lateral distance in most of the sensory areas of the brain
\cite{constantinidisCorrelatedDischargesPutative2002,rothschildFunctionalOrganizationPopulation2010,cohenMeasuringInterpretingNeuronal2011,smithSpatialTemporalScales2008a,smithSpatialTemporalScales2013a,denmanStructurePairwiseCorrelation2014}.
Nevertheless, there are functional and anatomical evidence,
that hint at deviations from these canonical features in PFC.
PFC is a central sub-network [in a graph-theoretic sense]  \cite{modhaNetworkArchitectureLongdistance2010} that play a crucial role in cognitive computations \citep{millerIntegrativeTheoryPrefrontal2001},
especially due to an increase in the integrative aspect of information processing in higher-order cortical areas.
Moreover, anatomical studies have shown that in contrast to early visual cortical areas
where we have a limited spread of lateral connections, in later stages of cortical hierarchy like PFC
\cite{amirCorticalHierarchyReflected1993,kritzerIntrinsicCircuitOrganization1995,angelucciCircuitsLocalGlobal2002,tanigawaOrganizationHorizontalAxons2005,vogesModelerViewSpatial2010}
lateral connections are considerably expanded 
\cite{levittTopographyPyramidalNeuron1993,amirCorticalHierarchyReflected1993,lundComparisonIntrinsicConnectivity1993,kritzerIntrinsicCircuitOrganization1995,fujitaIntrinsicConnectionsMacaque1996,tanigawaOrganizationHorizontalAxons2005}.
In this study, we investigate the functional connectivity ventro-lateral PFC (vlPFC) as a function of lateral distance.

\subsection*{Material and Methods}\label{sec:papers-pnas2018-material-methods}
In this study, we investigate the correlated fluctuations of single-neuron discharges in a mesoscopic scale.
Electrophysiology data was recorded from 4 macaque monkeys,
two in anesthetized state, and two in awake state.
Spiking activity was recorded from a Utah array chronically implanted in vlPFC.
For the awake experiments, monkeys were trained to fixate for 1000 ms on
moving grating in 8 different directions distributed randomly across multiple trials.
Tasks were started with the appearance of a red dot as a fixation point (with the size of $0.2^\circ$) on the screen for $\sim$300 ms (followed by a moving grating in one of the 8 directions).
The moving grating was only presented if the monkey maintains the fixation for the $\sim$300 ms period.
Moving grating had the size of $8^\circ$, speed of 12-13 degrees per second, and spatial frequency of 0.5 cycles per degree.

In anesthetized experiments, monkeys were exposed with 10 s of stimulation with natural movies.
Both awake and anesthetized experiments also included,
spontaneous sessions where neural activities recorded in the absence of any behavioral task.

Tuning curves were computed based on conventional procedures \cite{cohenMeasuringInterpretingNeuronal2011} by averaging the firing rate across trials for each of the eight presented directions of motion.
Signal correlations were defined as the correlation coefficient between the tuning curves of a neuronal pair.

Noise correlations for anesthetized data were computed by dividing the period of visual stimulation into 10 periods, each being 1000 ms long, and considered these periods as different successive stimuli.
The same procedure was used for the intertrial periods as well.
In the awake data, visual stimulation and intertrial periods were 1000 ms long each;
therefore, no additional procedure was required.
In the spontaneous data (both anesthetized and awake),
the entire length of the recording period was divided into periods of 1000 ms bins and they were treated as a trial.

The spike count correlation coefficients were computed similarly to previous classical studies
\cite{bairCorrelatedFiringMacaque2001a}
First, for each condition (either presentation of each moving grating in awake experiment or a single bin of movie clip in the anesthetized experiment),
we normalized the spike counts across all trials by converting them into z scores.
For each pair, we computed the Pearson's correlation coefficient for normalized spike counts and averaged across conditions to obtain the correlation value.


\subsection*{Results}

We found that the spatial structure of functional connectivity
(measured based on noise correlations) in vlPFC is different from most of the sensory cortices.
In most sensory cortices, noise correlations decay monotonically as a function of distance;
nevertheless, in vlPFC we observed in both anesthetized and
awake monkeys noise correlation rises again after an initial decay.
Moreover, we showed that the characteristic non-monotonic spatial
structure in vlPFC,
is pronounced with structured visual stimulation.


\subsection*{Conclusion}

Our results suggest that spatial inhomogeneities in the functional
architecture of the PFC arise from strong local and long-range lateral
interactions between neurons.
These characteristic patterns of interactions among PFC neurons lead
to a non-monotonic spatial structure of correlations in vlPFC.
Moreover, the mentioned spatial inhomogeneities are pronounced during structured
visual stimulation in the awake state which can be instrumental for
distributed information processing in PFC.

%%% Local Variables:
%%% mode: latex
%%% TeX-master: "../phdThesis_csb"
%%% End:


