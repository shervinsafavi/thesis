\ctparttext
{
  This part provides a general idea of this thesis.
  % We suggest an important approach toward understanding the brain,
  We suggest an important approach that should be taken toward understanding the brain,
  could be borrowed or inspired from the field of \emph{complex systems}.
  In light of this perspective, new questions can be asked in various domains 
  and moreover, old questions can be revisited based on this perspective. 
  Contents of this thesis, pertain to three different domains, namely
  \textit{methods for  neural data analysis}, \textit{neural theories}, and \textit{cognition}.
  % With this  motivation,
% 1
  In the first domain, 
  we introduce novel statistical methods for multi-scale investigation of neural data
  that we believe should be an important piece in our analysis methods for understanding the brain as a complex system.
% 2
  In the second domain, we first briefly introduce
  \textit{criticality hypothesis of the brain},
  % that has been borrowed from the field of complex systems
  that has been primarily developed based on statistical physics 
  and has been suggested to explain the complex dynamics of the brain activity in different spatial and temporal scales.
  Then we introduce our complementary approach of investigation in this framework,
  and our finding regarding the hypotheses.
  % investigate whether the brain is operating close to a critical is state or not.
  % Then we introduce its potential connection to \textit{efficient coding} 
  % which in contrast to criticality hypothesis of the brain 
  % is a functionally relevant 
  In the third domain,
  we first describe the importance of investigating bistable perception phenomenon from the perspective of complex systems.
  Then we discuss our finding pertaining the mesoscopic neural mechanism underlying this phenomenon.
  % introduce the question we asked and their answers.
  % With this  motivation, we introduce number of questions in three different domains, 
  % \textit{methods for analyis of neural data}, \textit{neural dynamics}, and \textit{behavior and cognition}.
  % We explain our three 
  % apporaches toward understaning the brain 
  % from the perspective of approaching the brain as a complex system.
} % Text on the Part 1 page describing the content in Part 2

%%% Local Variables:
%%% mode: latex
%%% TeX-master: "../phdThesis_csb"
%%% End:
