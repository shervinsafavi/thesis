\chapter{Paper \rom{7}}\label{cha:paper-kapoor2020}
\section*{Paper information} % 

\begin{description}
\item[Title:]
  Decoding the contents of consciousness from prefrontal ensembles
\item[Authors:]
  Vishal Kapoor$^*$,
  Abhilash Dwarakanath$^*$,
  Shervin Safavi,
  Joachim Werner,
  Michel Besserve,
  Theofanis I. Panagiotaropoulos,
  Nikos K. Logothetis
  ($^*$ indicate equal contributions) 
\item[Status:]
  Accepted for publication in Nature Communication (preprint is available online, see \citet{kapoorDecodingInternallyGenerated2022})
\item[Presentation at scientific meetings:]
  FFRM 2015 \cite{antoniouPerceptualModulationPupillary2015a},
  SfN 2018 \cite{panagiotaropoulosModulationNeuralDischarges2018a},
  FENS 2018 \cite{kapoorSpikingActivityPrefrontal2018},
  ASSC 2019 \cite{kapoorNeuronalDischargesPrefrontal2019}
\item[Author contributions: ]
  V.K., A.D. and T.I.P. designed the study. V.K., A.D. and S.S. trained animals. V.K. and A.D.
  performed experiments and collected data, with occasional help from S.S. V.K. and A.D.
  analyzed the data. S.S. contributed to spike sorting and selectivity analysis of control
  experiments. M.B. contributed to the decoding analysis. V.K. prepared and arranged the figures
  in the final format. S.S. provided the MATLAB generated version of the figures displayed in
  figure 3D, S12, S13 and S14 A. T.I.P. and N.K.L. supervised the study. N.K.L. and J.W.
  contributed unpublished reagents/analytical tools. N.K.L. provided the support to the group. V.K.
  and T.I.P. wrote the original manuscript draft. All authors participated in discussion and
  interpretation of the results and editing the manuscript.
\end{description}

%%% Local Variables:
%%% mode: latex
%%% TeX-master: "../phdThesis_csb"
%%% End:


\section*{Summary} %
\subsection*{Motivation}
The role of prefrontal cortex (PFC) has been controversial in recent consciousness studies.
Different frameworks of consciousness attribute different, even contradictory roles for PFC in generation of conscious experience.
Several frameworks, namely,
frontal lobe hypothesis \cite{crickConsciousnessNeuroscience1998},
higher order theory \cite{lauEmpiricalSupportHigherorder2011} and
global neuronal workspace framework \cite{baarsGlobalWorkspaceTheory2005,dehaeneExperimentalTheoreticalApproaches2011}
consider PFC play a mechanistic role in generation of conscious experience.
On the opposite side, another important framework of studying consciousness,
integrated theory of consciousness \cite{tononiInformationIntegrationTheory2004,balduzziIntegratedInformationDiscrete2008b,balduzziQualiaGeometryIntegrated2009b,oizumiPhenomenologyMechanismsConsciousness2014}
(for a review see \citet{tononiIntegratedInformationTheory2016}),
does not consider a similar role for PFC in generation of conscious experience,
rather attribute the role of PFC to prerequisites and consequences of  consciousness
\cite{aruDistillingNeuralCorrelates2012,degraafCorrelatesNeuralCorrelates2012}.

There are various differences between the aforementioned studies that support each of the two hypothesis.
For instance, studies that support attributing the role of PFC to prerequisites and consequences of consciousness,
used fMRI as the primary measurement technique, which can potentially lead to discrepancies.
In contrast, studies that support the opposite conclusion use electrophysiology
(see \autoref{cha:paper-safavi2014} for a short discussion).
Second, a large portion of studies that support a mechanistic role for PFC in conscious perception,
use externally induced perceptual switches such as Binocular Flash Suppression (BFS) \cite{panagiotaropoulosNeuronalDischargesGamma2012}.
Third, the majority of the experiments used behavioral reports by the subject in order to know the content of conscious experience (for a review see \cite{tsuchiyaNoReportParadigmsExtracting2015,kochNeuralCorrelatesConsciousness2016}).
This study was an effort, to bring this controversy one step closer to the resolution by
recording the neural activity from monkey ventro-lateral PFC (vlPFC) during a no-report Binocular Rivalry (BR) paradigm.

Focus of investigations on phenomenon of BR, in terms of spatio-temporal scales of measurements,
was mainly micro-scale (level of individual neurons) and macro-scale (level of large-scale networks)
Almost all the previous studies either focus on the activity of feature selective neurons measured based on single unit recordings
\cite{lehkyNoBinocularRivalry1996,sheinbergRoleTemporalCortical1997,kelirisRolePrimaryVisual2010,bahmaniNeuralCorrelatesBinocular2011,panagiotaropoulosNeuronalDischargesGamma2012},
or the whole-brain dynamics measured with imaging techniques (EEG/MEG, fMRI)
\cite{wangBrainMechanismsSimple2013,lumerNeuralCorrelatesPerceptual1998,srinivasanIncreasedSynchronizationNeuromagnetic1999b,hippOscillatorySynchronizationLargescale2011a,doesburgRhythmsConsciousnessBinocular2009b,tononiNeuralCorrelatesConsciousness2008,imamogluChangesFunctionalConnectivity2012}
(for reviews see \cite{blakeVisualCompetition2002a,panagiotaropoulosSubjectiveVisualPerception2014a,kochNeuralCorrelatesConsciousness2016}).
A complex system perspective to binocular rivalry phenomenon, motivates observation of the system in a mesoscopic scale as a very first step to understand the role of neural interactions (see \autoref{sec:why-it-appealing} for further elaboration).
In this study, we address this need, by measuring spiking activity of neural populations in vlPFC with multi-electrode recording techniques.


\subsection*{Material and Methods}\label{sec:papers-biorxiv2020-material-methods}
In this study, we investigate the neural correlate of visual awareness in mesoscopic scale.
Recording procedure is similar to awake experiments of \nameref{cha:paper-safavi2018} explained earlier (see \matmet of \autoref{sec:papers-pnas2018-material-methods}).
The core behavioral paradigm used in this study was a passive ambiguous stimulation, 
and consist of two tasks, Binocular Rivalry (BR) and Physical Alternation (PA).
Both tasks consist of fixation period similar to fixation task explained earlier in \autoref{sec:papers-pnas2018-material-methods},
and followed by presentation of 1 or 2 seconds  upward or downward moving gratings
(presented only to one eye -- half of the trials for each eye).
After the phase of stimulus presentation,
in PA trials, the first stimulus was removed and a moving grating in the contralateral eye was presented in the opposite direction.
BR trials had the identical structure of the stimulus presentation,
but with the difference that, the second stimulus was presented without removing the first stimulus.
In BR trials that two opposite moving grating were presented simultaneously, 
the perception of the monkey spontaneously switches between the stimulus
(\ie upward and downward grating) across the the entire length of trial (8-10 seconds).
Whereas, in PA trials, there are no perceptual switches,  but perception of the animal changes by the alternation of the presented stimuli (upward and downward grating).
Parameters of the visual stimulus (moving gratings) are identical to the experiment explained in \autoref{sec:papers-pnas2018-material-methods}.
Furthermore, Optokinetic Nystagmus (OKN) reflexes
\footnote{
  OKN reflexes are characteristic patterns of eye movements in response to moving stimuli,
  that consist of smooth pursuit and fast saccadic eye movements.}
has been used to determine the perception of the animal.

In addition to the main experiment that consist of BR and PA tasks,
we additionally have a control experiment for controlling eye movement as a confounding factor.
Given that determining the animal perception is based on eye movements (OKN reflexes),
to rule out the eye movement as a confounding factor,
we perform a passive fixation experiment similar to the awake experiment of \nameref{cha:paper-safavi2018} explained earlier (see \matmet of \autoref{sec:papers-pnas2018-material-methods}), but without eye movement.
In this experiment, the eye movement during presentation of moving grating were suppressed by instructing the animal to maintain the fixation during the task
(by overlaying a fixation point with size of $1$-$2^\circ$ on top of the moving grating).

\subsection*{Results}
Firstly, the perpetual dominance periods detected based on OKN reflexes follow a gamma distribution which is compatible with previous studies \cite{leveltNoteDistributionDominance1967}.
This indicates that using no-report paradigms of BR lead to compatible results with human studies.
Given the availability of neurons [recorded by Utah array] that respond to direction of motion of moving grating stimuli in PFC (see \nameref{cha:paper-safavi2018}),
we can quantify the proportion of perceptual modulation of neurons in our experiment that use upward and downward moving gratings as rivaling patterns. 
Interestingly, compatible with previous studies that used different tasks and visual stimuli \cite{panagiotaropoulosNeuronalDischargesGamma2012},
majority of sensory modulated units were also perceptually modulated.
Moreover, in the population level, the content of conscious perception of the animals was decodable from spiking activity of neural populations in vlPFC.
Lastly, the decoding algorithm that we used for decoding the content of the perception \cite{meyersNeuralDecodingToolbox2013},
could also reliably decode the content of the presented visual stimulus
(in the passive fixation experiment) both in presence and absence of eye movement
\ie training the decoder with responses in presence of eye movement,
and test when the eye movement are suppressed (fixation-on task) and vice versa.
Therefore, our control analysis suggest that
eye movements are not a confounding factor for our perceptual modulation.


\subsection*{Conclusion}
In this study, we showed that activity of the majority of sensory modulated neurons of vlPFC is correlated with conscious perception in a no-report binocular rivalry task,
and the content of conscious experience is decodable from mesoscopic dynamics of PFC.
Moreover, this study has an important implication for the neural correlate of visual awareness.
This study adds another piece of evidence for the involvement of PFC in conscious perception
which has been an important debate in the field of consciousness research in the last few years (also see \nameref{cha:paper-safavi2014}).

%%% Local Variables:
%%% mode: latex
%%% TeX-master: "../phdThesis_csb"
%%% End:



