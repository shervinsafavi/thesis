\chapter{Approaching through cognition} \label{cha:appr-thro-behav}

As motivated in \autoref{cha:brain-as-complex},
one of the apertures for approaching the brain as a complex system,
that let us remain close to the neuroscience side, is 
through behavior and cognition.
% 
After providing a brief introduction to visual awareness and related phenomenon such as binocular rivalry, 
% 
we argue that,
binocular rivalry is one of the important cognitive phenomenon,
that is particularly relevant for a complex system perspective toward the brain.
Based on this perspective toward binocular rivalry,
we suggest and conduct novel experimental works.
We study the phenomena of binocular rivalry on a mesoscopic scale which has not been done before.


\section{Visual awareness} %
Consciousness is one of the most challenging problems of science
\cite{christofQuestConsciousnessNeurobiological2004}.
However, during the last few decades, the vast technological and theoretical advancements brought consciousness research to an intense experimental phase. 
As a result, philosophical speculations on the nature and mechanisms of consciousness are slowly being replaced by empirical and theoretical approaches \cite{logothetisVisionWindowConsciousness2006,tononiNeuralCorrelatesConsciousness2008,christofkochConsciousnessConfessionsRomantic2012}.

There are various experimental paradigms in studying conscious\-ness.
We mention two example approaches and highlight our choice.
The first one is studying brain activity during various levels of conscious\-ness, \ie the differences between an awake, conscious state and various degrees of unconscious\-ness such as deep sleep, anesthesia, or coma.
The second one is studying how brain activity changes when a specific visual stimulus is subject\-ively perceived or supp\-ressed through experim\-ental paradigms like Binocular Rivalry (BR), Binocular Flash Suppression (BFS), masking etc. 

The first branch is about studying how brain activity changes in concert with changes in the overall level of consciousness, and indeed it is a fundamental approach. 
Nevertheless, it is extremely complex and it imposes a set of theoretical and experimental limitations. 
For example, it is technically difficult to monitor intracortical electrophysiological activity under conditions of coma. 
However, the second approach, \ie studying visual awareness 
(a "visual form of consciousness" \cite{crickVisualPerceptionRivalry1996a}), 
is an alternative approach to the problem with a more tractable framework,
especially at the neuronal level. 
In this approach, brain activity is monitored during changes in the \emph{content of} consciousness. 
For example, electrophysiological activity is studied when a visual stimulus becomes visible or invisible,
while everything else, including the overall level of consciousness as well as the sensory input, 
remains as constant as possible. 
Therefore, investigating various kinds of brain activity and their relation with the perception-related events ultimately might bring us steps closer toward an understanding of the neural mechanisms involved in visual awareness.

\subsection{Binocular rivalry}
One prominent example of such experimental paradigms that have been exhaustively exploited for
understanding the neural mechanisms involved in visual awareness is binocular
rivalry.
Binocular rivalry is one of the forms of ambiguous visual stimulation. 
It involves simultaneous stimulation of corresponding retinal locations
across the two eyes with incongruent visual stimuli.
It has been shown that different species experience this kind of ambiguous stimulation with some common characteristic \cite{carterPerceptualRivalryAnimal2020}.
When the subjects are presented with such visual stimuli, 
they typically experience fluctuations in perception between the two visual stimuli
(these fluctuations in perception are known as perceptual switches).

\subsection{Neural correlate of binocular rivalry}
In order to understand the neural correlate of phenomenon of binocular rivalry,
brain activity can be measured using various experimental methodologies at different scales. 
It can be spike trains from an individual neuron, field potentials or hemodynamic signals that reflect groups of neurons etc. 
Each measurement technique has its own limitations \cite{logothetisWhatWeCan2008}. 
For instance, non-invasive brain-imaging techniques are limited by their spatial and/or temporal resolution,
and electrophysiological recordings are limited in their coverage of cell populations.
Although all have their own limitations, 
they have provided us with a significant set of ideas about the neural mechanisms involved in conscious visual perception
that we briefly review in the following
(for detailed reviews, see for example \citet{blakeVisualCompetition2002a,tononiNeuralCorrelatesConsciousness2008,panagiotaropoulosSubjectiveVisualPerception2014a,kochNeuralCorrelatesConsciousness2016}).


Through single-unit recordings, we grasped a significant set of ideas and insights about the neural mechanisms underlying conscious visual perception on a local scale.
Specifically, through these studies, we learned that within each stage of visual hierarchy (from Lateral Geniculate Nucleus, V1 all the way to Profrontal Cortex (PFC)) 
there are a number of single units whose activity reflects the content of subjective perception of the animal.
The proportion of neurons which are modulated by the perception of the animal gradually increases across the visual hierarchy \cite{panagiotaropoulosSubjectiveVisualPerception2014a}.
From no modulated cell in Lateral Geniculate Nucleues (LGN) \cite{lehkyNoBinocularRivalry1996},
to superior temporal sulcus (STS) and inferotemporal cortex (IT) \cite{sheinbergRoleTemporalCortical1997},
and Lateral Prefrontal Cortex (LPFC) \cite{panagiotaropoulosNeuronalDischargesGamma2012,kapoorDecodingInternallyGenerated2022}
where 60-90\% of feature selective neurons are perceptually modulated.
But how does the activity of these distributed neurons relate to each other and also to other neurons (that are not involved in perception)?
How do they interact within their own population?
How is the activity of neuronal populations and large-scale networks organized, and how  are they related to perception-related events?
Single unit studies have potentially overlooked these important aspects of the underlying neural mechanisms,
Perhaps, such information is hidden in dynamic patterns of activity that are distributed over larger populations of neurons.

On the other side, imaging studies to some degree characterized the global network by revealing some specific large-scale interactions. 
For example, frequency-specific oscillatory interactions in the fronto-parieto-occipital \cite{hippOscillatorySynchronizationLargescale2011a}
and prefrontal-parietal networks \cite{doesburgRhythmsConsciousnessBinocular2009b}
and causal interactions in prefrontal-occipital \cite{imamogluChangesFunctionalConnectivity2012} network are involved in conscious perception.
However, these findings could not capture the \emph{neuronal} interactions due to their limited spatial and/or temporal resolution.
Indeed, such information is potentially available to multi-electrode recordings.

\section{Why is appealing from a complex system perspective} \label{sec:why-it-appealing}
An integrationist overview on the previous electrophysiology and imaging studies on the neural mechanisms involved in conscious visual perception implies that \emph{a global network of neuronal populations that interact with each other is involved in this phenomenon} \cite{blakeVisualCompetition2002a,panagiotaropoulosSubjectiveVisualPerception2014a}.
Therefore, visual awareness presumably is  a system property,
which is associated with a set of cooperative interactions within and between highly interconnected networks of neurons. 
These neurons are distributed within the entire thalamo-cortical system, 
mainly temporal, prefrontal,  occipital, parietal, lobes and thalamus 
\cite{blakeVisualCompetition2002a,panagiotaropoulosSubjectiveVisualPerception2014a,wangBrainMechanismsSimple2013,lumerNeuralCorrelatesPerceptual1998,srinivasanIncreasedSynchronizationNeuromagnetic1999b,hippOscillatorySynchronizationLargescale2011a,doesburgRhythmsConsciousnessBinocular2009b,panagiotaropoulosNeuronalDischargesGamma2012,bahmaniNeuralCorrelatesBinocular2011,tononiNeuralCorrelatesConsciousness2008,kochNeuralCorrelatesConsciousness2016,imamogluChangesFunctionalConnectivity2012}. 
The fact that, there is a large number of \emph{interacting} components (neurons and brain regions) involved in the phenomenon of  visual awareness,
is already one of the important characteristics that 
allows us to conceive perception as an \emph{emergent} property of a complex system.

Given this new conceptualization for visual awareness,
what are our  options to tackle it experimentally -- at least in terms of measuring the brain activity?
Almost all the previous studies of binocular rivalry
--in terms of spatial and temporal resolution-- 
are either single-unit recordings or whole-brain imaging (EEG/MEG, fMRI).
Such measurements can provide hints or evidence for the existence of such a distributed network (as indeed have been profoundly insightful), 
but they are not the most suitable  measurement techniques to characterize the \emph{neural interactions}
\footnote{
  With EEG/MEG and fMRI we can also characterize the interaction between the component of the neural system,
  but due to the nature of these measurement techniques, the picture they can provide about neural interactions is more ambiguous compare to what we can get from invasive recording techniques}.
Understanding the \emph{interaction} between units of a complex system is the key for characterizing collective behaviors and therefore it is important to observe the system at scales which give the clearest picture in this regard.
At first glance, we can realize that the phenomenon of binocular rivalry is poorly understood at the mesoscopic scale, 
which could not only reveal the phenomenon of coordinated activity 
within areas but also across areas in large-scale networks 
(see \autoref{sec:necess-invest-across}).
Therefore, a complex system perspective motivates observation at the mesoscopic scale as the first priority
and therefore motivates new experiments.
Studying at this scale, not only can inform about the involved cooperative mechanisms,
but also, it is the first step for bridging the studies based on single-unit recordings and imaging studies.


Conceiving perception as a system property or an emergent property
resulting from interactions within a large and distributed network of neurons, 
is not the only reason for the glamour of binocular rivalry from a complex system perspective.
Indeed, various models based on the theory of the dynamical system
(which is one of the most powerful frameworks to formalize a complex system)
can explain a range of characteristics of bistable perception (such as the distribution of dominance periods)
\cite{ditzingerOscillationsPerceptionAmbiguous1989b,braunAttractorsNoiseTwin2010,theodoniCorticalMicrocircuitDynamics2011a,pastukhovMultistablePerceptionBalances2013a}.
Perhaps, the most  appealing theoretical explanation is provided by \citet{pastukhovMultistablePerceptionBalances2013a}
that showed a network model operating on the edge of a bifurcation and can explain statistical characteristics of a wide range of multi-stable phenomenon.

Overall, based on available empirical and theoretical evidence we know, we need to deal with a large and distributed network of neurons;
Components of this network interact in a non-trivial way;
Phenomenon of binocular rivalry seems to be inherently multi-scale;
It seems, a neural network operating on an edge of bifurcation can explain various behavior-related statistical properties of the phenomena. 
Altogether, these finding make this phenomenon appealing from the perspective of complex systems.
We believe one of the very first steps for understating the cooperative neural mechanism pertaining to
binocular rivalry is \emph{measuring the mesoscopic neural activity},
\ie new experiments are needed which is the focus of the next sections.

\section{Experimental considerations} \label{sec:exper-cons}
In the previous section (\autoref{sec:why-it-appealing}) we argued that
meso-scale observations are necessary for understating the binocular rivalry  and
consequently, conducting new experiments are needed.
For conducting the  experimental work pertaining to binocular rivalry,
in addition to considerations pertaining to the level of observation,
some basic factors need to be considered as well.
These factors are briefly discussed in this section.

The first consideration pertains the recording area.
One of the target regions for new experiments is PFC for multiple reasons.
First, 
PFC is a central subnetwork (in a graph-theoretic sense)  \citep{modhaNetworkArchitectureLongdistance2010} that play a crucial role in cognitive computations \citep{millerIntegrativeTheoryPrefrontal2001},
especially due to an increase in the integrative aspect of information processing in higher-order cortical areas.
Second, ventro-lateral PFC (vlPFC), is reciprocally connected to Inferior Temporal (IT) coretex,
which contains the largest proportion of neurons that are perceptually modulated \citep{sheinbergRoleTemporalCortical1997} and neurons in PFC have been also shown to be perceptually modulated in similar tasks \citep{panagiotaropoulosNeuronalDischargesGamma2012,hesseNewNoreportParadigm2020}.
Third, PFC is outside of the core visual hierarchy.

For recording from PFC, we also need to be cautious with experimental design,
due to the ambiguous role of PFC in perception.
In a study by \citet{frassleBinocularRivalryFrontal2014},
it was suggested that ``frontal areas are associated with active report and introspection rather than with rivalry per se.''.
In \citet{safaviFrontalLobeInvolved2014} (\seealso, \nameref{cha:paper-safavi2014}), based on a broad set of evidence,
we argue that evidence provided by \citet{frassleBinocularRivalryFrontal2014} is not sufficient for this conclusion, and understating the role of PFC in visual awareness needs further investigation.
Due to potential confounding in activity of PFC that can happen due to behavioral report,
we needed to employ a no-report paradigm (decoding the perception of the animal using optokinetic nystagmus (OKN) responses \citep{leopoldMeasuringSubjectiveVisual2003}).

In this experiment, we particularly  needed to have the responses of neurons whose activities are modulated by features of a presented visual stimulus, 
and the visual stimulus had to induce OKN responses
(a certain pattern of eye movement in response to moving stimuli such as moving grating).
At the same time, as the core idea was monitoring the activity of neural population,
the recording had to be performed with Utah array 
(10 x 10 array of electrodes that need to be implanted chronically).
In contrast to previous similar experiments (\eg see \citet{panagiotaropoulosNeuronalDischargesGamma2012}) that used non-chronic recording with tetrods where the experimenter could explore to find the neuron by moving the electrodes,
Utah arrays are fixed and almost permanent.
In \citet{safaviNonmonotonicSpatialStructure2018} and \citet{kapoorDecodingInternallyGenerated2022}
(\seealsos, \nameref{cha:paper-safavi2018} and \nameref{cha:paper-kapoor2020})
we reported that such neurons are accessible with this recording technique (recording with Utah arrays) and under our experimental design.
Additionally, we also found that, similarly tuned neurons in this region of PFC
are correlated in large distances \cite{safaviNonmonotonicSpatialStructure2018} in contrast to most of sensory cortices
\cite{rothschildFunctionalOrganizationPopulation2010,cohenMeasuringInterpretingNeuronal2011,smithSpatialTemporalScales2008a,smithSpatialTemporalScales2013a,denmanStructurePairwiseCorrelation2014} (but also see \cite{rosenbaumSpatialStructureCorrelated2017}).
Interestingly, we also found that spatial structure of functional connectivity in ventro-lateral PFC is generally
\footnote{By generally, it is meant in presence and absence of visual stimulation, in awake and anesthetized state of the animal.}
different from  most sensory cortices.
In most  sensory cortices, noise correlation decay monotonically as a function of distance,
nevertheless, in ventro-lateral PFC we observed in both anesthetized and awake monkeys noise correlation rises again after an initial decay.
This observation is also compatible with anatomical differences between PFC and sensory areas
\cite{levittTopographyPyramidalNeuron1993,amirCorticalHierarchyReflected1993,lundComparisonIntrinsicConnectivity1993,kritzerIntrinsicCircuitOrganization1995,fujitaIntrinsicConnectionsMacaque1996,tanigawaOrganizationHorizontalAxons2005}.
The finding on the spatial structure of noise correlation in vlPFC was not relevant for the binocular rivalry experiment as the spatial structures were not take into account,
nevertheless, it was an important finding of the circuitry of PFC.

\section{Toward a meso-scale understanding}
The very first question that can be approached based on a mesosopic-level investigation,
is  what can population dynamics  reflect about the content of conscious perception.
Second question is what can we learn about the involved neural mechanism from micro-meso relationships in PFC.
Notably, both questions are approachable when we have observed the system in a mesoscopic scale
(level of neural populations), and are briefly discussed in the next sections (and associated papers).

\subsection{Meso-scale dynamics} %

The activity of the majority of PFC neurons that are responsive to visual attributes of sensory input
are correlated with conscious perception of animals as well.
In our case, we used vertically moving grating  -- upward or downward as stimuli \cite{safaviNonmonotonicSpatialStructure2018,kapoorDecodingInternallyGenerated2022}
(\seealsos, \nameref{cha:paper-safavi2018} and \nameref{cha:paper-kapoor2020})
and previously it was shown this is the case for face-selective neurons as well \cite{panagiotaropoulosNeuronalDischargesGamma2012}.
But additionally, the content of conscious perception is decodable from the spiking activity of neural \emph{populations} in ventro-lateral PFC.
This is the first confirmation of informativity of the meso-scale observation or measurement of the neural activity.
The next steps should focus on characterizing the coordinated dynamics and neural interactions 
(see the next section and the \autoref{part:outlook} for further elaboration on the next steps).


\subsection{Micro-Meso relationship} % 
Given the empirical evidence on the informativeness of population spiking of PFC neurons,
more specifically the fact that they reflect the content of conscious perception,
it is justified to consider more intricate aspects of mesoscopic dynamics.
Such aspect of mesoscopic dynamics includes signatures of neural coordination such as neural oscillation and spike-LFP relationship 
(also see \autoref{cha:appr-thro-nda} important aspect of neural coordination).
Furthermore, investigating the relationship between PFC [presumed] state fluctuations conjectured based on LFP oscillatory dynamics,
perceptual switches and spiking activity can hint at another aspect of the putative role of neural interactions in binocular rivalry.
Indeed, one of the important findings of our study was that,
spiking activity of population reflecting the dominant perception,
are coupled (relatively stronger than suppressed population) to LFP in range $25-45$ $Hz$ after the perceptual switch \cite{dwarakanathBistabilityPrefrontalStates2023}
(\seealso, \nameref{cha:paper-dwarakanath2020}).

This strong spike-LFP coupling can be a hint for an emphasized communication (or interaction) of PFC populations reflecting the conscious perception and other brain regions
(see \citet{buzsakiWhatDoesGamma2015} for the interpretation of spike-LFP coupling as a quantity to characterize the communication channel).
Further investigation is needed to characterize the interaction and functional role of this putative communication.
In particular, multiple experimental evidence should be taken into account for interpreting the functional role of the mentioned neuronal interaction.
First, we know that neural populations that monitor task-related activity exist in the same region of PFC in the absence of any behavioral report \citep{kapoorParallelFunctionallySegregated2018},
which is important given than various studies argue that PFC is strongly involved in task monitoring \cite{frassleBinocularRivalryFrontal2014}.
Second, we know that the activity of neural populations in IT cortex is also correlated with perception in the absence of behavioral reports \cite{hesseNewNoreportParadigm2020}.
On the other side, from studies with causal intervention, we know that the activity of PFC is needed for difficult object recognition tasks \cite{karFastRecurrentProcessing2020}.
Therefore, IT cortex might be a crucial component in this communication circuit and needed to be clarified in future studies.

%%% Local Variables:
%%% mode: latex
%%% TeX-master: "../phdThesis_csb"
%%% End:
