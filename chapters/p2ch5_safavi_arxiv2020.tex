\chapter{Paper \rom{1}} 
\label{cha:paper-5}
\section*{Paper information} % 

\begin{description}
\item[Title:]
  From univariate to multivariate coupling between continuous signals and point processes: A mathematical framework
\item[Authors:]
  Shervin Safavi, Nikos K. Logothetis, Michel Besserve
\item[Status:] 
  Published in Neural Computation, see \citet{safaviUnivariateMultivariateCoupling2021}
\item[Presentation at scientific meetings:]
  NeurIPS 2019 Workshop: Learning with Temporal Point Processes
  \cite{safaviMultivariateCouplingEstimation2019},
  Bernstein 2021 \cite{safaviGeneralizedPhaseLocking2021}  
\item[Author contributions: ]

  Conceptualization, S.S., and M.B.;
  Methodology, S.S., and M.B.;
  Software, S.S. and M.B.;
  Formal Analysis, S.S., and M.B.;
  Investigation, S.S., and M.B.;
  Resources, N.K.L.;
  Data Curation, S.S., and M.B.;
  Writing -- Original Draft, S.S., and M.B.;
  Writing -- Review \& Editing: S.S., M.B., and N.K.L.;
  Visualization, S.S., and M.B.;
  Supervision and Project administration, M.B.;
  Funding acquisition, N.K.L.
  
\end{description}


%%% Local Variables:
%%% mode: latex
%%% TeX-master: "../phdThesis_csb"
%%% End:


\section*{Summary} %
\subsection*{Motivation}
In various complex systems,
we deal with highly multi-variate temporal point processes,
that are corresponding to the activity of a large number of individuals.
They can be generated by the activity of neurons in brain networks \cite{johnsonPointProcessModels1996},
such as neurons' action potentials,
or by members in social networks \citep{daiRecurrentCoevolutionaryLatent2016,deLearningForecastingOpinion2016},
such as tweets in the Twitter network.
In practice, a limited number of events per unit are accessible experimentally or observable
(for instance numbers of spikes generated by neurons).
With such limitations, inferring the underlying dynamical properties of the studied system becomes challenging.
Nevertheless, in many cases, exploiting the coupling between the point processes and aggregate measure of the complex system (such as Local Field Potentials as an aggregate measure of population neural activity) can be insightful for understanding the underlying dynamics.

Meaningful and reliable estimates of coupling between such signals can be crucial for understanding many complex systems.
However, the statistical properties of many methods classically used remain poorly understood.
As a consequence, statistical assessment in practice largely relies on heuristics (\eg permutation tests).
While such approaches often make intuitive sense, they are computationally expensive and may be biased by properties of the data that are unaccounted for.
This is particularly relevant for quantities involving point processes and high-dimensional data, which have largely non-intuitive statistical properties, and yet are key tools for experimentalists and data analysts.
In this study, we establish a principled framework for statistical analysis of coupling between multi-variate point process and continuous signal.

\subsection*{Material and Methods}
First, we derive analytically the asymptotic distribution for a class of coupling statistics that quantify the correlation between a point process and a continuous signal.
The key to this theoretical prediction is expressing coupling statistics as stochastic integrals. 
Indeed, a general family of coupling measures can be expressed as stochastic integrals. 
The Martingale Central Limit Theorem allows us to derive analytically the asymptotic Gaussian distribution of such coupling measures. 
We show that these coupling statistics follow a Gaussian distribution. 
A commonly used example of such coupling statistics is Phase Locking Value (PLV) which typically is used for quantifying spike-LFP coupling in neuroscience.

We then go beyond uni-variate coupling measures and analyze the statistical properties of a family of multi-variate coupling measures taking the form of a matrix with stochastic integral coefficients.
We characterize the joint Gaussian asymptotic distribution of matrix coefficients,
and exploit Random Matrix Theory (RMT) principles to show that,
after appropriate normalization,
the spectral distribution of such large matrices under the null hypothesis
(absence of coupling between the point process and continuous signals),
follows approximately the Marchenko-Pastur law \citep{marcenkoDistributionEigenvaluesSets1967a}
\footnote{Referred paper \cite{marcenkoDistributionEigenvaluesSets1967a},
  is not written in English, but it is  the original publication.
  Reader can refer to \citet[Chapter
  2]{andersonIntroductionRandomMatrices2010} instead.}
(which is a well-characterized distribution in Random Matrix Theory),
while the magnitude of the largest singular value converges to a fixed value whose simple analytic expression depends only on the shape of the matrix.

\subsection*{Results}
We derive analytically the asymptotic distribution of Phase-Locking Value (PLV)
which is a coupling statistic conventionally used for quantifying the relationship between a pair of a point process (like spikes) and an oscillatory continuous signal (like LFPs).
We show that PLVs follow a Gaussian distribution with calculable mean and variance.

Based on the multi-variate extension,
we show how this result provides a fast and principled procedure to detect significant singular values of the coupling matrix, reflecting an actual dependency between the underlying signals.
This is of paramount importance for the analysis of empirical data given the ever-increasing dimensionality of datasets that need computationally efficient statistical tests.

\subsection*{Conclusion}
Our results not only construct a theoretical framework, which is valuable on its own
but also can have various applications for neural data analysis and beyond.
For instance, based on our theoretical framework we note realistic scenarios where the PLV can be a biased estimator of spike-LFP coupling, and in light of our framework, such biases can be treated. 


%%% Local Variables:
%%% mode: latex
%%% TeX-master: "../phdThesis_csb"
%%% End:


