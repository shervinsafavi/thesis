\chapter{Paper \rom{3}}\label{cha:paper-besserve2020ned}
\section*{Paper information} % 

\begin{description}
\item[Title:] The complex spectral structure of transient LFPs reveals subtle aspects of network coordination across scales and structures
\item[Authors:]
  Michel Besserve, Shervin Safavi, Bernhard Sch\"olkopf, Nikos Logothetis
\item[Status:]
  Work-in-progress; a preliminary  manuscript is available in the appendix,
  see \hyperref[pdf:besserve2020ned]{Paper 3}.
\item[Presentation at scientific meetings:]
  Machine Learning Summer School \cite{besservePracticalMachineLearning2016}
\item[Author contributions: ]

  Conceptualization, M.B. and N.K.L;
  Methodology, M.B. and S.S.;
  Software, S.S. and M.B.;
  Formal Analysis, M.B.;
  Investigation, S.S. and M.B.;
  Resources, B.S. and N.K.L.;
  Data Curation, M.B. and N.K.L;
  Writing - Original Draft, M.B. and S.S.;
  Writing - Review \& Editing: M.B., S.S., B.S. and N.K.L;
  Visualization, M.B. and S.S.;
  Supervision and Project administration,  M.B.;
  Funding acquisition, B.S. and N.K.L.

\end{description}


%%% Local Variables:
%%% mode: latex
%%% TeX-master: "../phdThesis_csb"
%%% End:


\section*{Summary} %
\subsection*{Motivation}
LFPs are intermediary signals, and as such, they
reflect a mesoscopic picture of the brain dynamics \cite{liljenstroemMesoscopicBrainDynamics2012}.
As LFPs are rich signals \cite{buzsakiOriginExtracellularFields2012,liljenstroemMesoscopicBrainDynamics2012,einevollModellingAnalysisLocal2013},
they can be a pivotal point for bringing the brain dynamics at different scales together. 
In particular, certain transient activities of LFPs reflect cooperative dynamics (we call them \emph{neural events}).
A prominent example of such neural events are sharp wave-ripples (SWRs),
and it has been observed they co-occur with well-coordinated activity at smaller scales (neurons and populations of neurons) \cite{csicsvariEnsemblePatternsHippocampal2000,csicsvariEnsemblePatternsHippocampal2000,olivaRoleHippocampalCA22016},
as well as larger scale (entire brain) \cite{logothetisHippocampalCorticalInteraction2012,karimiabadchiSpatiotemporalPatternsNeocortical2020}.
In spite of the importance of such characteristic neural activities (neural events),
there are not many principled methods for identifying them in a single channel LFP.
We introduce a principled method for identifying neural events in a single channel LFP.

\subsection*{Material and Methods}
We detect the neural events by isolating transient characteristic neural activities.
We first compute the spectrograms of the LFP signals by applying short-term Fourier transform (STFT) on LFPs in order to exploit the spectral content of the LFPs.
To identify the frequent transient neural activity with similar spectral content we apply non-negative Matrix Factorization (NMF).
Notably, due to scale-invariant nature of LFPs (similar to other extracellular field potential \cite{buzsakiOriginExtracellularFields2012}) \cite{freemanScalefreeNeocorticalDynamics2007a,heScalefreeBrainActivity2014},
we used Itakura-Saito divergence in the optimization procedure of NMF \cite{fevotteNonnegativeMatrixFactorization2009}
in order to avoid under-weighting of high-frequency components due to their low power in the spectrum.
The components result from NMF, provide the information on the spectral content of the neural events.
In order to temporally isolate the neural events and characterize their temporal profile, 
we apply a shift-invariant dictionary learning
(a modified version of dictionary learning provided by \citet{mailheShiftinvariantDictionaryLearning2008a}).
The latter step, allows us to temporally locate the neural events and also identify the time-domain profiles of events that their spectral content are characterized by the NMF step.

We demonstrate the capability of our method by identifying neural events and their brain-wide signatures in Hippocampus and LGN recorded from anesthetized monkeys.
Furthermore, in order to demonstrate that neural events have the potential of  relating the meso-scale dynamics even to cellular dynamics,
we investigate the neural events in the simulation of thalamocortical circuitry  developed by \citet{costaThalamocorticalNeuralMass2016} where allow us to access both meso-scale dynamics and also some level of cellular dynamics.
The simulation consists of neural mass models with two modules,
one for the thalamus and one for the cortex, and mimics the behavior of these circuits during different stages of sleep.

\subsection*{Results}
We developed a novel methodology for detecting neural events (transient cooperative neural activities) such as sharp wave-ripples.
With our method, neural events can be detected with minimal prior knowledge about the structure under study.
Namely, the spectral content is automatically identified by the method,
and various other attributes of neural events such as the number of neural event clusters  can also be identified by the method in an unsupervised fashion.

Furthermore, we demonstrate the capability of the method by identifying neural events in Hippocampus and LGN and also explore their brain-wide \emph{macro-scale} signatures using concurrent fMRI recordings from anesthetized monkeys.
The results suggest that similar to the previous study of \citet{logothetisHippocampalCorticalInteraction2012} that was focused on sharp wave-ripples,
the identified events in Hippocampus and LGN reflect a large-scale coordinated dynamics.
Indeed, this demonstrates the insightfulness of neural events for bridging the meso-scale and macro-scale brain dynamics.


Our results also suggest that neural events can be insightful for establishing a bridge between meso-scale and micro-scale brain dynamics, even at the cellular level.
We demonstrate this aspect, by investigating a simulation of the thalamocortical system developed by \citet{costaThalamocorticalNeuralMass2016}.
With our methodology, we identified different kinds of spindles in the activity of the thalamus module of the simulation,
and demonstrate that different events co-occur with characteristic activity patterns in the cellular variables (such as membrane potentials and ionic currents) of the simulation.

\subsection*{Conclusion}

With this method, we can find characteristic patterns of LFPs in an unsupervised fashion.
This methodology not only allows us to detect well established neural events such as SWRs in a principled fashion,
it also identifies characteristic patterns in a single channel LFP that have not been explored, and they can be insightful about cooperative and multi-scale dynamics of the brain.
Such patterns are potentially very special in the sense that,
they provide us a time window at which meso-scale dynamics are closely related to micro- and macro-scale dynamics.
In fact, as pointed out in \autoref{sec:necess-invest-across} and \autoref{sec:need-new-tools}, this is of paramount importance for bridging the scales of neural dynamics,
in particular when combined with GPLA introduced in \nameref{cha:paper-gpla} and NET-fMRI \cite{logothetisNeuralEventTriggeredFMRILargescale2014}.

%%% Local Variables:
%%% mode: latex
%%% TeX-master: "../phdThesis_csb"
%%% End:


