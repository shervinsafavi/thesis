\chapter{Paper \rom{8}}\label{cha:paper-dwarakanath2020}
\section*{Paper information} % 

\begin{description}
\item[Title:]
  Prefrontal state fluctuations gate access to consciousness
\item[Authors:]
  Abhilash Dwarakanath$^*$,
  Vishal Kapoor$^*$,
  Joachim Werner,
  Shervin Safavi
  Leonid A. Fedorov,
  Nikos K. Logothetis,
  Theofanis I. Panagiotaropoulos
  ($^*$ indicate equal contributions) 
\item[Status:]
  Preprint is available online, see \citet{dwarakanathBistabilityPrefrontalStates2023}
\item[Presentation at scientific meetings:]
  FFRM 2015 \cite{antoniouPerceptualModulationPupillary2015a},
  SfN 2018 \cite{panagiotaropoulosModulationNeuralDischarges2018a},
  AREADNE 2018 \cite{dwarakanathPerisynapticActivityPrefrontal2018}
\item[Author contributions: ]
  
  Conceptualisation: A.D., V.K., T.I.P. (lead), N.K.L.;
  Data curation: A.D. (lead), V.K. and J.W.;
  Formal analysis: A.D. (lead), V.K., J.W., L.A.F.;
  Funding acquisition: N.K.L.;
  Investigation: A.D. (equal), V.K. (equal), T.I.P. (supporting);
  Methodology: A.D. (equal), V.K. (equal), J.W. \& S.S. (supporting), T.I.P. (equal);
  Project administration: T.I.P.;
  Resources: J.W., N.K.L. (lead);
  Software: A.D. (lead), V.K., J.W., L.A.F. \& S.S. (supporting);
  Supervision: T.I.P.;
  Visualisation: A.D. (lead), T.I.P. (supporting);
  Writing -- original draft: A.D., T.I.P. (lead);
  Writing -- review \& editing: A.D., V.K., L.A.F., S.S., T.I.P. (lead), N.K.L.

\end{description}

%%% Local Variables:
%%% mode: latex
%%% TeX-master: "../phdThesis_csb"
%%% End:


\section*{Summary} %
\subsection*{Motivation}
In \autoref{sec:why-it-appealing} we elaborated on the motivations for studying the phenomenon of binocular rivalry (BR) in a mesoscopic scale
and in \nameref{cha:paper-kapoor2020} we showed that content of conscious experience is decodable from mesoscopic dynamics of PFC. %which is the
This was the first confirmation on the usefulness of the meso-scale observation.
This allows us to go one step further in studying the mesoscopic dynamics of PFC.
One of the most important markers of coordination in mesoscopic dynamics of the brain,
is neural oscillations \cite{buzsakiRhythmsBrain2011,buzsakiScalingBrainSize2013a}.
In this study we investigate oscillatory dynamics in ventro-lateral PFC (vlPFC) and its connection to conscious visual perception.

\subsection*{Material and Methods}
Most of the experimental details for this study was explained in summaries of the other papers
(\nameref{cha:paper-safavi2018}, and \nameref{cha:paper-kapoor2020}).
Recording procedure is similar to awake experiment of
\nameref{cha:paper-safavi2018} explained earlier (see \matmet of \autoref{sec:papers-pnas2018-material-methods}).
The behavioral paradigm used in this study is also explained earlier
(see \matmet of \autoref{sec:papers-biorxiv2020-material-methods}).
In this study, Continuous Wavelet Transform (CWT) \cite{mallatWaveletTourSignal1999} has been used to extract spectral content of LFPs
and Chronux toolbox \cite{bokilChronuxPlatformAnalyzing2010} for quantifying spike-LFP coupling by computing Spike-Field-Coherence (SFC).


\subsection*{Results}
This study reveals various characteristic oscillatory activities which are happening in the vicinity of the perceptual switches detected based on Optokinetic Nystagmus (OKN) reflexes.
The frequency of these transient oscillatory activities are covering low and intermediate ranges (namely 1-9 Hz and 20-40 Hz).
In addition to presence of these coordinated dynamics in the mesoscopic activity of PFC neural populations and their relationship to perceptual events,
the statistics and spatio-temporal patterns of some of these transitory events lend support to important frameworks of studying the consciousness.

\subsection*{Conclusion}
This study adds another piece of evidence for the involvement of PFC in conscious perception, in addition to the one discussed earlier in \nameref{cha:paper-kapoor2020}.
In particular, it reveals signatures of neural coordination reflected in the oscillatory dynamics (see \autoref{sec:relat-betw-meso}) of neural populations involved in conscious visual perception.
Revealing these signatures could not be possible without investigating the system in meso-scale 
(see more elaborating in \autoref{sec:why-it-appealing}).
Lastly similar to \nameref{cha:paper-kapoor2020}, this study has an important implication for the neural correlate of visual awareness.
This study highlights the involvement of PFC in conscious perception
which has been an important debate in the field of consciousness research in the last few years (also see \nameref{cha:paper-safavi2014}).


%%% Local Variables:
%%% mode: latex
%%% TeX-master: "../phdThesis_csb"
%%% End:


