\chapter{Paper \rom{5}} \label{cha:paper-safavi2014}
\section*{Paper information} % 

\begin{description}
\item[Title:]
  Is the frontal lobe involved in conscious perception?
\item[Authors:]
  Shervin Safavi$^*$,
  Vishal Kapoor$^*$,
  Nikos K. Logothetis,
  Theofanis I. Panagiotaropoulos
  ($^*$ indicate equal contribution) 
\item[Status:]
  Published in Frontiers in Psychology, see \citet{safaviFrontalLobeInvolved2014}
\item[Author contributions: ]
  Conceptualization, S.S., V.K., N.K.L. and T.I.P.; 
  Methodology, not applicable;
  Software, not applicable;
  Formal Analysis, not applicable;
  Investigation, S.S., V.K. and T.I.P.; 
  Resources, N.K.L.;
  Data Curation, not applicable;
  Writing -- Original Draft, S.S., V.K. and T.I.P.; 
  Writing -- Review \& Editing, S.S., V.K., N.K.L. and T.I.P.; 
  Visualization, not applicable;
  Supervision and Project administration, T.I.P.;
  Funding acquisition, N.K.L.
\end{description}

%%% Local Variables:
%%% mode: latex
%%% TeX-master: "../phdThesis_csb"
%%% End:


\section*{Summary}


PFC as part of the subsystem that serves the goal-directed character of behavior 
\cite{logothetisStudiesLargeScaleNetworks2014},
needs to closely interact with two other subsystems. One is responsible for sensory representation and the other reflects the internal states of the organism,
such as arousal or motivation \cite{logothetisStudiesLargeScaleNetworks2014}.
Moreover, PFC is also a central sub-network [in a graph-theoretic sense]  \cite{modhaNetworkArchitectureLongdistance2010} that plays a crucial role in various cognitive functions \cite{millerIntegrativeTheoryPrefrontal2001}.
Therefore, it is expected to behave differently compared to sensory-related networks in various tasks (\eg binocular rivalry).

In recent years, novel paradigms have been used to dissociate the activity related to
conscious perception from the activity reflecting its prerequisites and consequences \cite{aruDistillingNeuralCorrelates2012,degraafCorrelatesNeuralCorrelates2012,tsuchiyaNoReportParadigmsExtracting2015}.
In particular, one of these studies focused on resolving the role of frontal lobe in conscious perception \cite{frassleBinocularRivalryFrontal2014}.
In this study, \citet{frassleBinocularRivalryFrontal2014} through a novel experimental design,
concluded that
``frontal areas are associated with active
report and introspection rather than with
rivalry per se.''
Therefore, activity in prefrontal regions could be considered as a consequence rather than a neural correlate of conscious perception.

However, based on both fMRI and electrophysiological studies we suspect that PFC is indeed involved in conscious visual perception.
Regarding the fMRI studies, \citet{zaretskayaIntrospectionAttentionAwareness2014}, in response to \citet{frassleBinocularRivalryFrontal2014},
reviewed the experimental evidence based on fMRI BOLD activity in frontal lobe which suggests even with contrastive analysis (similar to \citet{frassleBinocularRivalryFrontal2014}), some regions of frontal lobe are engaged and therefore play a role in conscious perception.
Electrophysiological studies  also provided evidence on involvement of some regions of frontal lobe in the absence of behavioral reports (\ie using no-report paradigms),
namely lateral PFC, in visual awareness \cite{panagiotaropoulosNeuronalDischargesGamma2012,kapoorDecodingInternallyGenerated2022,dwarakanathBistabilityPrefrontalStates2023}.
In particular, two recent studies  \cite{kapoorDecodingInternallyGenerated2022,dwarakanathBistabilityPrefrontalStates2023},
(which were carried out as a part of this thesis, see \autoref{cha:appr-thro-behav})
used a similar paradigm to the one used in \citet{frassleBinocularRivalryFrontal2014}.
Moreover, a recent study by \citet{kapoorParallelFunctionallySegregated2018}
based on analysis of a wider range of single units in vlPFC (not just feature selective neurons)
suggests that, both task-related and perception-related neurons co-exist in the same region of PFC.

Last but not least, the last decade witnessed a similar disagreement but on the role of primary visual cortex instead of frontal lobe
\cite{leopoldActivityChangesEarly1996,maierDivergenceFMRINeural2008a,kelirisRolePrimaryVisual2010,leopoldPrimaryVisualCortex2012}.
Ultimately, measuring both electrophysiological activity and the BOLD signal in the
same macaques engaged in an identical
task of perceptual suppression settled the debate \cite{maierDivergenceFMRINeural2008a,leopoldPrimaryVisualCortex2012}.
Therefore, to address such discrepancies we can benefit from multiple measurement techniques simultaneously or in the same animal along with a careful experimental design.

In this opinion paper, we advocate that 
formulating our conclusions related to prerequisites, consequences and true correlates of conscious experiences,
we need to have an \emph{integrative} view on the in hand collection of new evidence.
Our investigations and conclusions about the neural correlates of
consciousness must not only entail better designed experiments
but also diverse experimental techniques (e.g., BOLD fMRI, electrophysiology)
that could measure brain activity at different spatial
and temporal scales.
Moreover, different measurement techniques can reflect complementary information on the brain activity.
Therefor, such a multi-modal approach holds great promise in refining our current
understanding of conscious processing (and understating the brain in a broader sense).


%%% Local Variables:
%%% mode: latex
%%% TeX-master: "../phdThesis_csb"
%%% End:


