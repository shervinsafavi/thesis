\chapter{Paper \rom{4}}\label{cha:paper-safavi2020cribay}
\section*{Paper information} % 

\begin{description}
\item[Title:]
  Signatures of criticality in efficient coding networks
\item[Authors:]
  Shervin Safavi,
  Matthew Chalk,
  Nikos K. Logothetis,
  Anna Levina

\item[Status:]
  Work-in-progress; a preliminary  manuscript is available in the appendix,
  see \hyperref[pdf:safavi2020sce]{Paper 4}.
\item[Presentation at scientific meetings:]
  Conference on Complex Systems (CCS 2018) Satellite: Complexity from Cells to Consciousness: Free Energy, Integrated Information, and Epsilon Machines
  \cite{safaviOptimalEfficientCoding2018},
  DPG-Fr\"uhjahrstagung 2019
  \cite{safaviSignaturesCriticalityEfficient2019a},
  Cosyne 2020
  \cite{levinaSignaturesCriticalityObserved2020a}  
\item[Author contributions: ]

  Conceptualization, S.S., and A.L.;
  Methodology, S.S., M.C., A.L.;
  Software, S.S. and M.C;
  Formal Analysis, S.S., M.C and A.L.;
  Investigation, S.S., M.C and A.L.; 
  Resources, N.K.L. and A.L.;
  Data Curation, S.S., M.C and A.L.;
  Writing -- Original Draft, S.S.;
  Writing -- Review \& Editing, not applicable
  (this letter has not been communicated with other co-authors so far);
  Visualization, S.S.;
  Supervision and Project administration, A.L.;
  Funding acquisition, N.K.L. and A.L.
\end{description}

%%% Local Variables:
%%% mode: latex
%%% TeX-master: "../phdThesis_csb"
%%% End:


\section*{Summary} %
\subsection*{Motivation}

Understanding the computations that the brain needs to implement (neural computation)
and the dynamics of the brain activity (neural dynamics) are two important goals of computational neuroscience \cite[Chapter 1]{churchlandComputationalBrain1992}.
Ideally, we need a framework that can accommodate both aspects of the brain in one framework
\cite{churchlandComputationalBrain1992,eurichNeuralDynamicsNeural2003}.
Nevertheless, to the best of my knowledge, no framework has been developed to satisfy this important need.

An intermediate step toward developing such a framework is exploiting the frameworks and models that are either centered around neural computation or neural dynamics \emph{with implications for the other aspect}.
Indeed, there are normative models that have implications for neural dynamics
\cite{lengyelMatchingStorageRecall2005,deneveBayesianSpikingNeurons2008a,deneveBayesianSpikingNeurons2008,tanakaRecurrentInfomaxGenerates2008,buesingNeuralDynamicsSampling2011,boerlinPredictiveCodingDynamical2013,billDistributedBayesianComputation2015,chalkNeuralOscillationsSignature2016,zeldenrustEfficientRobustCoding2019,echevesteCorticallikeDynamicsRecurrent2020}
and also models of neural dynamics with implications for neural computation
\cite{bertschingerRealtimeComputationEdge2004,eliasmithUnifiedApproachBuilding2005,sussilloNeuralCircuitsComputational2014,hidalgoInformationbasedFitnessEmergence2014a,shrikiOptimalInformationRepresentation2016,maassSearchingPrinciplesBrain2016,kimLearningRecurrentDynamics2018,chenComputingModulatingSpontaneous2019,michielsvankessenichPatternRecognitionNeuronal2019,finlinsonOptimalControlExcitable2020}
We suggest seeking for ``bridges'' between such frameworks can be a first step.
%
Neural coding is of particular interest for building such bridges
as there have been various studies that suggest potential connections between neural coding and neural dynamics 
\cite{ermentroutRelatingNeuralDynamics2007c,boerlinPredictiveCodingDynamical2013,shrikiOptimalInformationRepresentation2016,chalkNeuralOscillationsSignature2016,alamiaAlphaOscillationsTraveling2019,kadmonPredictiveCodingBalanced2020,roethEfficientPopulationCoding2020,echevesteCorticallikeDynamicsRecurrent2020}.
In particular, multiple recent studies provide qualitative or quantitative evidence on the usefulness of operating close to a phase transition for coding
\cite{shrikiOptimalInformationRepresentation2016,chalkNeuralOscillationsSignature2016,kadmonPredictiveCodingBalanced2020,roethEfficientPopulationCoding2020}.
Interestingly, the phase transition is also one of the pillars of the criticality hypothesis of the brain
\cite{munozColloquiumCriticalityDynamical2018,tkacikInformationProcessingLiving2016,moraAreBiologicalSystems2011a}.
In spite of this apparent and exciting connection,
networks implementing neural coding have never been investigated for signatures of criticality.
In this study, we investigate networks that can be optimized for neural coding for signatures of criticality.

\subsection*{Material and Methods}

In this study, we investigate a network of Leaky-Integrate and Fire (LIF) neurons whose connectivity and dynamics can be optimized for coding a one-dimensional sensory input \cite{chalkNeuralOscillationsSignature2016}.
This network can be optimized to encode the input efficiently
(\ie with a minimal number of spikes) and accurately (\ie with minimal reconstruction error).
The input is reconstructed by performing a linear readout of spike trains
(see \cite{boerlinPredictiveCodingDynamical2013}).
Given an idealized network with instantaneous synapses, the optimal network could be derived analytically from first principles \cite{boerlinPredictiveCodingDynamical2013}.
In this case, neurons that receive a common input avoid communicating redundant information via instantaneous recurrent inhibition.
However, adding realistic synaptic delays leads to network synchronization, which impairs coding efficiency.
\citet{chalkNeuralOscillationsSignature2016} demonstrated that, in the presence of synaptic delays, a network of LIF neurons can nonetheless be optimized for efficient coding by adding noise to the network.
The network's performance depends non-monotonically on the noise amplitude,
with the optimal performance achieved for an intermediate noise level. 
% 
We investigate potential signatures of criticality such as the scale-freeness of neuronal avalanches \cite{beggsNeuronalAvalanchesNeocortical2003} in the spiking activity of the network.


\subsection*{Results}

In this study, we introduce a new approach to better connect neural dynamics and neural computation.
Here we search for a potential connection between models of neural dynamics with implications on neural computation,
and normative models of neural computation with implications for neural dynamics.
We search for signatures of criticality in neuronal networks that can be optimized based on objectives of efficient coding. 
We investigate
efficient coding networks for signatures of criticality.
Interestingly, almost exclusively in the optimized network, we observe the signatures of criticality
and when the noise amplitude is too low or too high for efficient coding, the network appears either super-critical or sub-critical, respectively.
In both cases, the noise level that was optimal for coding also resulted in a scale-free avalanche behavior.

\subsection*{Conclusion}

Our results suggest that coding-based optimality might co-occur with closeness to criticality.
This result has important implications, as it shows how two influential,
and previously disparate fields --- efficient coding, and criticality --- might be intimately related.
This work proposes several promising avenues for future research on the computation and dynamics of the neural system.


%%% Local Variables:
%%% mode: latex
%%% TeX-master: "../phdThesis_csb"
%%% End:


