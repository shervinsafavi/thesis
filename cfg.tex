%%%%%%%%%%%%%%%%%%%%%%%%%%%%%%%%%%%%%%%%%
% Thesis Configuration File
%
% The main lines to change in this file are in the DOCUMENT VARIABLES
% section, the rest of the file is for advanced configuration.
%
%%%%%%%%%%%%%%%%%%%%%%%%%%%%%%%%%%%%%%%%%

%----------------------------------------------------------------------------------------
%	DOCUMENT VARIABLES
%	Fill in the lines below to enter your information into the thesis template
%	Each of the commands can be cited anywhere in the thesis
%----------------------------------------------------------------------------------------

% Remove drafting to get rid of the '[ Date - classicthesis version 4.0 ]' text at the bottom of every page
\PassOptionsToPackage{eulerchapternumbers,listings, pdfspacing, subfig,beramono,eulermath,parts}{classicthesis}
% Available options: drafting parts nochapters linedheaders eulerchapternumbers beramono eulermath pdfspacing minionprospacing tocaligned dottedtoc manychapters listings floatperchapter subfig
% Adding 'dottedtoc' will make page numbers in the table of contents flushed right with dots leading to them

\newcommand{\myTitle}{Brain as a Complex System}
\newcommand{\mySubtitle}{harnessing systems neuroscience tools \& notions for an empirical approach}

\newcommand{\myName}{Shervin safavi\xspace}
\newcommand{\myProf}{Dr.\ Nikos Logothetis\xspace}
\newcommand{\myOtherProf}{Dr.\ Michel Besserve\xspace}
\newcommand{\myReader}{Prof.\ Stefano Panzeri\xspace}

\newcommand{\myInst}{Max Planck Institute for Biological Cybernetics\xspace}
\newcommand{\myInstS}{Max Planck Institute for Intelligent Systems\xspace}
\newcommand{\myInstDept}{Dept. of Physiology of Cognitive Processes\xspace}
\newcommand{\myInstDeptLong}{Department of Physiology of Cognitive Processes\xspace}
\newcommand{\myInstDeptS}{Dept. of Empirical Inference\xspace}
\newcommand{\myInstDeptSlong}{Department of Empirical Inference\xspace}
\newcommand{\myUni}{Eberhard Karls University of T\"ubingen\xspace}
\newcommand{\ScFac}{Faculty of Science\xspace}
\newcommand{\MedFac}{Faculty of Medicine\xspace}

\newcommand{\myIMPRS}{IMPRS for Cognitive and Systems Neuroscience\xspace}
\newcommand{\GTC}{Graduate Training Centre of Neuroscience\xspace}
\newcommand{\NIP}{School of Neural Information Processing\xspace}
\newcommand{\myFaculty}{where\xspace}

\newcommand{\myLocation}{T\"ubingen, Germany\xspace}
\newcommand{\myTime}{December 2019\xspace}
\newcommand{\myVersion}{Version 0.0\xspace}

%----------------------------------------------------------------------------------------
%	USEFUL COMMANDS
%----------------------------------------------------------------------------------------

\newcommand{\ie}{i.\,e.\xspace}
\newcommand{\Ie}{I.\,e.\xspace}
\newcommand{\eg}{e.\,g.\xspace}
\newcommand{\Eg}{E.\,g.\xspace}
\newcommand{\seealso}{also can refer to the corresponding summary}
\newcommand{\seealsos}{also can refer to the corresponding summaries}

\makeatletter
\newcommand*{\rom}[1]{\textsf{\expandafter\@slowromancap #1@}}
\makeatother
\newcommand{\matmet}{``Material and Methods''}

\newcounter{dummy} % Necessary for correct hyperlinks (to index, bib, etc.)
\providecommand{\mLyX}{L\kern-.1667em\lower.25em\hbox{Y}\kern-.125emX\@}

% separattor line, based on
% https://tex.stackexchange.com/questions/130762/prl-style-horizontal-line-in-latex
\newcommand{\PRLsep}{\noindent\makebox[\linewidth]{\resizebox{0.3333\linewidth}{1pt}{$\bullet$}}\bigskip}


%----------------------------------------------------------------------------------------
%	MATH/PHYSICS USEFUL COMMANDS
%----------------------------------------------------------------------------------------
\newcommand{\avg}[1]{\langle #1 \rangle} % for average
\newcommand{\abs}[1]{\left| #1 \right|} % for absolute value

%----------------------------------------------------------------------------------------
%	PACKAGES
%----------------------------------------------------------------------------------------

\usepackage{lipsum} % Used for inserting dummy 'Lorem ipsum' text into the template

%------------------------------------------------
 
\PassOptionsToPackage{latin9}{inputenc} % latin9 (ISO-8859-9) = latin1+"Euro sign"
\usepackage{inputenc}
 
 %------------------------------------------------

\PassOptionsToPackage{ngerman,american}{babel}  % Change this to your language(s)
% Spanish languages need extra options in order to work with this template
%\PassOptionsToPackage{spanish,es-lcroman}{babel}
\usepackage{babel}

%------------------------------------------------			

% \PassOptionsToPackage{super, comma}{natbib}
\PassOptionsToPackage{comma,square,numbers}{natbib}
\newcommand\citeA[1]{\citeauthor{#1}}
\newcommand\citeY[1]{(\citeyear{#1})}
\newcommand\citeAYt[1]{\citeauthor{#1} (\citeyear{#1})}
\newcommand\citeAYp[1]{(\citeauthor{#1} \citeyear{#1})}
\newcommand\citeAYpwop[2]{(\citeauthor{#1} \citeyear[#2]{#1})} % wop = with options

% \usepackage{natbib}

\usepackage[			% use biblatex for bibliography
	backend=biber,      % use biber or bibtex backend
        style=numeric,   % choose style
        citestyle=numeric-comp,
	natbib=true,		% allow natbib commands
	hyperref=true,	    % activate hyperref support
	alldates=year,      % only show year (not month)
        mincitenames=1,
        maxcitenames=1,
        backref=true,
        doi=true,
        url=false,
        isbn=false,
        eprint=false,
        sorting=none
        ]{biblatex}


% Suppress "In:" for articles (by Herbert);
% from https://tex.stackexchange.com/questions/10682/suppress-in-biblatex
\renewbibmacro{in:}{}

        
\IfFileExists{/home/ssafavi/Nextcloud/libraries/zotlib.bib}{\bibliography{/home/ssafavi/Nextcloud/libraries/zotlib,locallib,tmplib}}{\bibliography{locallib,tmplib}}

% % bib source setting (this might need to go to a seperate bib file)
% \IfFileExists{/home/ssafavi/Nextcloud/libraries/zotlib.bib}{\bibliography{/home/ssafavi/Nextcloud/libraries/zotlib.bib,locallib,tmplib.bib}}{\bibliography{locallib.bib,tmplib.bib}}%,tmplib.bib}}
        
 %------------------------------------------------

\PassOptionsToPackage{fleqn}{amsmath} % Math environments and more by the AMS 
 \usepackage{amsmath}
 
 %------------------------------------------------

\PassOptionsToPackage{T1}{fontenc} % T2A for cyrillics
\usepackage{fontenc}

%------------------------------------------------

\usepackage{xspace} % To get the spacing after macros right

%------------------------------------------------

\usepackage{mparhack} % To get marginpar right

%------------------------------------------------

\usepackage{fixltx2e} % Fixes some LaTeX stuff 

%------------------------------------------------

\PassOptionsToPackage{smaller}{acronym} % Include printonlyused in the first bracket to only show acronyms used in the text
\usepackage{acronym} % nice macros for handling all acronyms in the thesis

%------------------------------------------------

%\renewcommand*{\acsfont}[1]{\textssc{#1}} % For MinionPro
% \renewcommand{\bflabel}[1]{{#1}\hfill} % Fix the list of acronyms

%------------------------------------------------

\PassOptionsToPackage{pdftex}{graphicx}
\usepackage{graphicx} 

%----------------------------------------------------------------------------------------
%	FLOATS: TABLES, FIGURES AND CAPTIONS SETUP
%----------------------------------------------------------------------------------------

\usepackage{tabularx} % Better tables
\setlength{\extrarowheight}{3pt} % Increase table row height
\newcommand{\tableheadline}[1]{\multicolumn{1}{c}{\spacedlowsmallcaps{#1}}}
\newcommand{\myfloatalign}{\centering} % To be used with each float for alignment
\usepackage{caption}
\captionsetup{format=hang,font=small}
\usepackage{subfig}  
\usepackage{epstopdf} % eps conversion

%----------------------------------------------------------------------------------------
%	CODE LISTINGS SETUP
%----------------------------------------------------------------------------------------

\usepackage{listings} 
\lstset{language=[LaTeX]Tex, % Specify the language for listings here
keywordstyle=\color{RoyalBlue}, % Add \bfseries for bold
basicstyle=\small\ttfamily, % Makes listings a smaller font size and a different font
commentstyle=\color{Green}\ttfamily, % Color of comments
stringstyle=\rmfamily, % Font type to use for strings
numbers=left, % Change left to none to remove line numbers
numberstyle=\scriptsize, % Font size of the line numbers
stepnumber=5, % Increment of line numbers
numbersep=8pt, % Distance of line numbers from code listing
showstringspaces=false, % Sets whether spaces in strings should appear underlined
breaklines=true, % Force the code to stay in the confines of the listing box
frame=single, % Frame border - none/leftline/topline/bottomline/lines/single/shadowbox/L
belowcaptionskip=.75\baselineskip % Space after the "Listing #: Desciption" text and the listing box
}

%----------------------------------------------------------------------------------------
%	CUSTOMIZED color
%----------------------------------------------------------------------------------------

\usepackage[dvipsnames]{xcolor}
\definecolor{grayForLinks}{gray}{0.40} %  %
\definecolor{webgreen}{rgb}{0,.5,0}

%----------------------------------------------------------------------------------------
%	HYPERREFERENCES
%----------------------------------------------------------------------------------------

\PassOptionsToPackage{pdftex,hyperfootnotes=false,pdfpagelabels}{hyperref}
\usepackage{hyperref}  % backref linktocpage pagebackref
\pdfcompresslevel=9
\pdfadjustspacing=1

\hypersetup{
% Uncomment the line below to remove all links (to references, figures, tables, etc)
%draft, 
colorlinks=true, linktocpage=true, pdfstartpage=1, pdfstartview=Fit,
% Uncomment the line below if you want to have black links (e.g. for printing black and white)
%colorlinks=false, linktocpage=false, pdfborder={0 0 0}, pdfstartpage=3, pdfstartview=FitV, 
breaklinks=true, pdfpagemode=UseNone, pageanchor=true, pdfpagemode=UseOutlines,
plainpages=false, bookmarksnumbered, bookmarksopen=true, bookmarksopenlevel=1,
hypertexnames=true, pdfhighlight=/O, urlcolor=grayForLinks, linkcolor=RoyalBlue, citecolor=webgreen,
%------------------------------------------------
% PDF file meta-information
pdftitle={\myTitle},
pdfauthor={\textcopyright\ \myName, \myInst},
pdfsubject={},
pdfkeywords={},
pdfcreator={pdfLaTeX},
pdfproducer={LaTeX with hyperref and classicthesis}
%------------------------------------------------
}   

%----------------------------------------------------------------------------------------
%	BACKREFERENCES
%----------------------------------------------------------------------------------------

\usepackage{ifthen} % Allows the user of the \ifthenelse command
% \newboolean{enable-backrefs} % Variable to enable backrefs in the bibliography
% \setboolean{enable-backrefs}{true} % Variable value: true or false

% \newcommand{\backrefnotcitedstring}{\relax} % (Not cited.)
% \newcommand{\backrefcitedsinglestring}[1]{(Cited on page~#1.)}
% \newcommand{\backrefcitedmultistring}[1]{(Cited on pages~#1.)}
% \ifthenelse{\boolean{enable-backrefs}} % If backrefs were enabled
% {
% \PassOptionsToPackage{hyperpageref}{backref}
% \usepackage{backref} % to be loaded after hyperref package 
% \renewcommand{\backreftwosep}{ and~} % separate 2 pages
% \renewcommand{\backreflastsep}{, and~} % separate last of longer list
% \renewcommand*{\backref}[1]{}  % disable standard
% \renewcommand*{\backrefalt}[4]{% detailed backref
% \ifcase #1 
% \backrefnotcitedstring
% \or
% \backrefcitedsinglestring{#2}
% \else
% \backrefcitedmultistring{#2}
% \fi}
% }{\relax} 

%----------------------------------------------------------------------------------------
%	AUTOREFERENCES SETUP
%	Redefines how references in text are prefaced for different 
%	languages (e.g. "Section 1.2" or "section 1.2")
%----------------------------------------------------------------------------------------

\makeatletter
\@ifpackageloaded{babel}
{
\addto\extrasamerican{
\renewcommand*{\figureautorefname}{Figure}
\renewcommand*{\tableautorefname}{Table}
\renewcommand*{\partautorefname}{Part}
\renewcommand*{\chapterautorefname}{Chapter}
\renewcommand*{\sectionautorefname}{Section}
\renewcommand*{\subsectionautorefname}{Section}
\renewcommand*{\subsubsectionautorefname}{Section}
}
\addto\extrasngerman{
\renewcommand*{\paragraphautorefname}{Absatz}
\renewcommand*{\subparagraphautorefname}{Unterabsatz}
\renewcommand*{\footnoteautorefname}{Fu\"snote}
\renewcommand*{\FancyVerbLineautorefname}{Zeile}
\renewcommand*{\theoremautorefname}{Theorem}
\renewcommand*{\appendixautorefname}{Anhang}
\renewcommand*{\equationautorefname}{Gleichung}
\renewcommand*{\itemautorefname}{Punkt}
}
\providecommand{\subfigureautorefname}{\figureautorefname} % Fix to getting autorefs for subfigures right
}{\relax}
\makeatother

%----------------------------------------------------------------------------------------

\usepackage{classicthesis} 

%----------------------------------------------------------------------------------------
%	ShS SPECIFIC PACKAGES
%----------------------------------------------------------------------------------------

\usepackage{comment}
\numberwithin{equation}{chapter}
\numberwithin{figure}{chapter}

\usepackage{pdfpages} % for adding PDF files

\usepackage{nameref}

% \usepackage{dirtytalk}
\usepackage{csquotes}

% for adding chemical stuff
\usepackage{chemmacros}
\chemsetup[phases]{pos=sub}

%----------------------------------------------------------------------------------------
%	ShS SPECIFIC settings
%----------------------------------------------------------------------------------------

% \setlength{\parskip}{10pt}       % to increase the space between paragraphs
\usepackage{animate}             % for animations
\def \fmisc {./gfx/} % address for addiional figures
\usepackage{microtype}             


%----------------------------------------------------------------------------------------
% additional packages
\usepackage{bibentry}


%----------------------------------------------------------------------------------------
%	ShS COLORS
%----------------------------------------------------------------------------------------

\definecolor{pantone328}{cmyk}{1,0,0.57,0.30}
\newcommand{\shs}[1]{\textcolor{red}{[\scriptsize \textit{shervin}: #1]}}

%%% Local Variables:
%%% mode: latex
%%% TeX-master: "phdThesis_csb"
%%% End:

\colorlet{CTtitle}{pantone328}
\colorlet{CTurl}{grayForLinks}

%----------------------------------------------------------------------------------------
%	CHANGING TEXT AREA 
%----------------------------------------------------------------------------------------

% \setlength{\parindent}{0cm}
% \setlength{\parindent}{1.5em}
%\linespread{1.05} % a bit more for Palatino
%\areaset[current]{312pt}{761pt} % 686 (factor 2.2) + 33 head + 42 head \the\footskip
% \setlength{\marginparwidth}{7em}%
\setlength{\marginparwidth}{7.05em}%
% \setlength{\marginparsep}{2em}%

%----------------------------------------------------------------------------------------
%	USING DIFFERENT FONTS
%----------------------------------------------------------------------------------------

% \usepackage[oldstylenums]{kpfonts} % oldstyle notextcomp
% \usepackage[osf]{libertine}
% \usepackage{hfoldsty} % Computer Modern with osf
% \usepackage[light,condensed,math]{iwona}
% \renewcommand{\sfdefault}{iwona}
% \usepackage{lmodern} % <-- no osf support :-(
% \usepackage[urw-garamond]{mathdesign} % <-- no osf support :-(

%%% Local Variables:
%%% mode: latex
%%% TeX-master: "phdThesis_csb"
%%% TeX-master: "phdThesis_csb"
%%% End:
